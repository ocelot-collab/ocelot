% Generated by Sphinx.
\def\sphinxdocclass{report}
\documentclass[letterpaper,10pt,english]{sphinxmanual}
\usepackage[utf8]{inputenc}
\DeclareUnicodeCharacter{00A0}{\nobreakspace}
\usepackage{cmap}
\usepackage[T1]{fontenc}
\usepackage{babel}
\usepackage{times}
\usepackage[Bjarne]{fncychap}
\usepackage{longtable}
\usepackage{sphinx}
\usepackage{multirow}


\title{OCELOT Documentation}
\date{July 20, 2016}
\release{16.7}
\author{IA}
\newcommand{\sphinxlogo}{}
\renewcommand{\releasename}{Release}
\makeindex

\makeatletter
\def\PYG@reset{\let\PYG@it=\relax \let\PYG@bf=\relax%
    \let\PYG@ul=\relax \let\PYG@tc=\relax%
    \let\PYG@bc=\relax \let\PYG@ff=\relax}
\def\PYG@tok#1{\csname PYG@tok@#1\endcsname}
\def\PYG@toks#1+{\ifx\relax#1\empty\else%
    \PYG@tok{#1}\expandafter\PYG@toks\fi}
\def\PYG@do#1{\PYG@bc{\PYG@tc{\PYG@ul{%
    \PYG@it{\PYG@bf{\PYG@ff{#1}}}}}}}
\def\PYG#1#2{\PYG@reset\PYG@toks#1+\relax+\PYG@do{#2}}

\expandafter\def\csname PYG@tok@gd\endcsname{\def\PYG@tc##1{\textcolor[rgb]{0.63,0.00,0.00}{##1}}}
\expandafter\def\csname PYG@tok@gu\endcsname{\let\PYG@bf=\textbf\def\PYG@tc##1{\textcolor[rgb]{0.50,0.00,0.50}{##1}}}
\expandafter\def\csname PYG@tok@gt\endcsname{\def\PYG@tc##1{\textcolor[rgb]{0.00,0.27,0.87}{##1}}}
\expandafter\def\csname PYG@tok@gs\endcsname{\let\PYG@bf=\textbf}
\expandafter\def\csname PYG@tok@gr\endcsname{\def\PYG@tc##1{\textcolor[rgb]{1.00,0.00,0.00}{##1}}}
\expandafter\def\csname PYG@tok@cm\endcsname{\let\PYG@it=\textit\def\PYG@tc##1{\textcolor[rgb]{0.25,0.50,0.56}{##1}}}
\expandafter\def\csname PYG@tok@vg\endcsname{\def\PYG@tc##1{\textcolor[rgb]{0.73,0.38,0.84}{##1}}}
\expandafter\def\csname PYG@tok@m\endcsname{\def\PYG@tc##1{\textcolor[rgb]{0.13,0.50,0.31}{##1}}}
\expandafter\def\csname PYG@tok@mh\endcsname{\def\PYG@tc##1{\textcolor[rgb]{0.13,0.50,0.31}{##1}}}
\expandafter\def\csname PYG@tok@cs\endcsname{\def\PYG@tc##1{\textcolor[rgb]{0.25,0.50,0.56}{##1}}\def\PYG@bc##1{\setlength{\fboxsep}{0pt}\colorbox[rgb]{1.00,0.94,0.94}{\strut ##1}}}
\expandafter\def\csname PYG@tok@ge\endcsname{\let\PYG@it=\textit}
\expandafter\def\csname PYG@tok@vc\endcsname{\def\PYG@tc##1{\textcolor[rgb]{0.73,0.38,0.84}{##1}}}
\expandafter\def\csname PYG@tok@il\endcsname{\def\PYG@tc##1{\textcolor[rgb]{0.13,0.50,0.31}{##1}}}
\expandafter\def\csname PYG@tok@go\endcsname{\def\PYG@tc##1{\textcolor[rgb]{0.20,0.20,0.20}{##1}}}
\expandafter\def\csname PYG@tok@cp\endcsname{\def\PYG@tc##1{\textcolor[rgb]{0.00,0.44,0.13}{##1}}}
\expandafter\def\csname PYG@tok@gi\endcsname{\def\PYG@tc##1{\textcolor[rgb]{0.00,0.63,0.00}{##1}}}
\expandafter\def\csname PYG@tok@gh\endcsname{\let\PYG@bf=\textbf\def\PYG@tc##1{\textcolor[rgb]{0.00,0.00,0.50}{##1}}}
\expandafter\def\csname PYG@tok@ni\endcsname{\let\PYG@bf=\textbf\def\PYG@tc##1{\textcolor[rgb]{0.84,0.33,0.22}{##1}}}
\expandafter\def\csname PYG@tok@nl\endcsname{\let\PYG@bf=\textbf\def\PYG@tc##1{\textcolor[rgb]{0.00,0.13,0.44}{##1}}}
\expandafter\def\csname PYG@tok@nn\endcsname{\let\PYG@bf=\textbf\def\PYG@tc##1{\textcolor[rgb]{0.05,0.52,0.71}{##1}}}
\expandafter\def\csname PYG@tok@no\endcsname{\def\PYG@tc##1{\textcolor[rgb]{0.38,0.68,0.84}{##1}}}
\expandafter\def\csname PYG@tok@na\endcsname{\def\PYG@tc##1{\textcolor[rgb]{0.25,0.44,0.63}{##1}}}
\expandafter\def\csname PYG@tok@nb\endcsname{\def\PYG@tc##1{\textcolor[rgb]{0.00,0.44,0.13}{##1}}}
\expandafter\def\csname PYG@tok@nc\endcsname{\let\PYG@bf=\textbf\def\PYG@tc##1{\textcolor[rgb]{0.05,0.52,0.71}{##1}}}
\expandafter\def\csname PYG@tok@nd\endcsname{\let\PYG@bf=\textbf\def\PYG@tc##1{\textcolor[rgb]{0.33,0.33,0.33}{##1}}}
\expandafter\def\csname PYG@tok@ne\endcsname{\def\PYG@tc##1{\textcolor[rgb]{0.00,0.44,0.13}{##1}}}
\expandafter\def\csname PYG@tok@nf\endcsname{\def\PYG@tc##1{\textcolor[rgb]{0.02,0.16,0.49}{##1}}}
\expandafter\def\csname PYG@tok@si\endcsname{\let\PYG@it=\textit\def\PYG@tc##1{\textcolor[rgb]{0.44,0.63,0.82}{##1}}}
\expandafter\def\csname PYG@tok@s2\endcsname{\def\PYG@tc##1{\textcolor[rgb]{0.25,0.44,0.63}{##1}}}
\expandafter\def\csname PYG@tok@vi\endcsname{\def\PYG@tc##1{\textcolor[rgb]{0.73,0.38,0.84}{##1}}}
\expandafter\def\csname PYG@tok@nt\endcsname{\let\PYG@bf=\textbf\def\PYG@tc##1{\textcolor[rgb]{0.02,0.16,0.45}{##1}}}
\expandafter\def\csname PYG@tok@nv\endcsname{\def\PYG@tc##1{\textcolor[rgb]{0.73,0.38,0.84}{##1}}}
\expandafter\def\csname PYG@tok@s1\endcsname{\def\PYG@tc##1{\textcolor[rgb]{0.25,0.44,0.63}{##1}}}
\expandafter\def\csname PYG@tok@gp\endcsname{\let\PYG@bf=\textbf\def\PYG@tc##1{\textcolor[rgb]{0.78,0.36,0.04}{##1}}}
\expandafter\def\csname PYG@tok@sh\endcsname{\def\PYG@tc##1{\textcolor[rgb]{0.25,0.44,0.63}{##1}}}
\expandafter\def\csname PYG@tok@ow\endcsname{\let\PYG@bf=\textbf\def\PYG@tc##1{\textcolor[rgb]{0.00,0.44,0.13}{##1}}}
\expandafter\def\csname PYG@tok@sx\endcsname{\def\PYG@tc##1{\textcolor[rgb]{0.78,0.36,0.04}{##1}}}
\expandafter\def\csname PYG@tok@bp\endcsname{\def\PYG@tc##1{\textcolor[rgb]{0.00,0.44,0.13}{##1}}}
\expandafter\def\csname PYG@tok@c1\endcsname{\let\PYG@it=\textit\def\PYG@tc##1{\textcolor[rgb]{0.25,0.50,0.56}{##1}}}
\expandafter\def\csname PYG@tok@kc\endcsname{\let\PYG@bf=\textbf\def\PYG@tc##1{\textcolor[rgb]{0.00,0.44,0.13}{##1}}}
\expandafter\def\csname PYG@tok@c\endcsname{\let\PYG@it=\textit\def\PYG@tc##1{\textcolor[rgb]{0.25,0.50,0.56}{##1}}}
\expandafter\def\csname PYG@tok@mf\endcsname{\def\PYG@tc##1{\textcolor[rgb]{0.13,0.50,0.31}{##1}}}
\expandafter\def\csname PYG@tok@err\endcsname{\def\PYG@bc##1{\setlength{\fboxsep}{0pt}\fcolorbox[rgb]{1.00,0.00,0.00}{1,1,1}{\strut ##1}}}
\expandafter\def\csname PYG@tok@kd\endcsname{\let\PYG@bf=\textbf\def\PYG@tc##1{\textcolor[rgb]{0.00,0.44,0.13}{##1}}}
\expandafter\def\csname PYG@tok@ss\endcsname{\def\PYG@tc##1{\textcolor[rgb]{0.32,0.47,0.09}{##1}}}
\expandafter\def\csname PYG@tok@sr\endcsname{\def\PYG@tc##1{\textcolor[rgb]{0.14,0.33,0.53}{##1}}}
\expandafter\def\csname PYG@tok@mo\endcsname{\def\PYG@tc##1{\textcolor[rgb]{0.13,0.50,0.31}{##1}}}
\expandafter\def\csname PYG@tok@mi\endcsname{\def\PYG@tc##1{\textcolor[rgb]{0.13,0.50,0.31}{##1}}}
\expandafter\def\csname PYG@tok@kn\endcsname{\let\PYG@bf=\textbf\def\PYG@tc##1{\textcolor[rgb]{0.00,0.44,0.13}{##1}}}
\expandafter\def\csname PYG@tok@o\endcsname{\def\PYG@tc##1{\textcolor[rgb]{0.40,0.40,0.40}{##1}}}
\expandafter\def\csname PYG@tok@kr\endcsname{\let\PYG@bf=\textbf\def\PYG@tc##1{\textcolor[rgb]{0.00,0.44,0.13}{##1}}}
\expandafter\def\csname PYG@tok@s\endcsname{\def\PYG@tc##1{\textcolor[rgb]{0.25,0.44,0.63}{##1}}}
\expandafter\def\csname PYG@tok@kp\endcsname{\def\PYG@tc##1{\textcolor[rgb]{0.00,0.44,0.13}{##1}}}
\expandafter\def\csname PYG@tok@w\endcsname{\def\PYG@tc##1{\textcolor[rgb]{0.73,0.73,0.73}{##1}}}
\expandafter\def\csname PYG@tok@kt\endcsname{\def\PYG@tc##1{\textcolor[rgb]{0.56,0.13,0.00}{##1}}}
\expandafter\def\csname PYG@tok@sc\endcsname{\def\PYG@tc##1{\textcolor[rgb]{0.25,0.44,0.63}{##1}}}
\expandafter\def\csname PYG@tok@sb\endcsname{\def\PYG@tc##1{\textcolor[rgb]{0.25,0.44,0.63}{##1}}}
\expandafter\def\csname PYG@tok@k\endcsname{\let\PYG@bf=\textbf\def\PYG@tc##1{\textcolor[rgb]{0.00,0.44,0.13}{##1}}}
\expandafter\def\csname PYG@tok@se\endcsname{\let\PYG@bf=\textbf\def\PYG@tc##1{\textcolor[rgb]{0.25,0.44,0.63}{##1}}}
\expandafter\def\csname PYG@tok@sd\endcsname{\let\PYG@it=\textit\def\PYG@tc##1{\textcolor[rgb]{0.25,0.44,0.63}{##1}}}

\def\PYGZbs{\char`\\}
\def\PYGZus{\char`\_}
\def\PYGZob{\char`\{}
\def\PYGZcb{\char`\}}
\def\PYGZca{\char`\^}
\def\PYGZam{\char`\&}
\def\PYGZlt{\char`\<}
\def\PYGZgt{\char`\>}
\def\PYGZsh{\char`\#}
\def\PYGZpc{\char`\%}
\def\PYGZdl{\char`\$}
\def\PYGZhy{\char`\-}
\def\PYGZsq{\char`\'}
\def\PYGZdq{\char`\"}
\def\PYGZti{\char`\~}
% for compatibility with earlier versions
\def\PYGZat{@}
\def\PYGZlb{[}
\def\PYGZrb{]}
\makeatother

\begin{document}

\maketitle
\tableofcontents
\phantomsection\label{index::doc}



\chapter{Overview}
\label{index:overview}\label{index:ocelot-documentation}
OCELOT is a framework

cpbd module

adaptors

Installation notes


\chapter{Contents:}
\label{index:contents}

\section{Charged Particle Beam Dynamics (CPBD) module}
\label{cpbd:charged-particle-beam-dynamics-cpbd-module}\label{cpbd::doc}

\subsection{Overview}
\label{cpbd:overview}
Charged Particle Beam Dynamics module provides features for charged particle (electron) beam optics, including
calculating and matching Twiss parameters, single-particle tracking as well as tracking with collective effects (CSR, space charge and wakefields)


\subsection{Getting started}
\label{cpbd:getting-started}
Import OCELOT

\begin{Verbatim}[commandchars=\\\{\}]
     \PYG{k+kn}{from} \PYG{n+nn}{ocelot} \PYG{k+kn}{import} \PYG{o}{*}
\end{Verbatim}

Define a magnetic lattice

\begin{Verbatim}[commandchars=\\\{\}]
     \PYG{n}{q1} \PYG{o}{=} \PYG{n}{Quadrupole}\PYG{p}{(}\PYG{n}{l} \PYG{o}{=} \PYG{l+m+mf}{0.3}\PYG{p}{,} \PYG{n}{k1} \PYG{o}{=} \PYG{l+m+mi}{5}\PYG{p}{)}
     \PYG{n}{q2} \PYG{o}{=} \PYG{n}{Quadrupole}\PYG{p}{(}\PYG{n}{l} \PYG{o}{=} \PYG{l+m+mf}{0.3}\PYG{p}{,} \PYG{n}{k1} \PYG{o}{=} \PYG{o}{\PYGZhy{}}\PYG{l+m+mi}{5}\PYG{p}{)}
     \PYG{n}{d} \PYG{o}{=} \PYG{n}{Drift}\PYG{p}{(}\PYG{n}{l} \PYG{o}{=} \PYG{l+m+mf}{0.5}\PYG{p}{)}
     \PYG{n}{lat} \PYG{o}{=} \PYG{n}{MagneticLattice}\PYG{p}{(} \PYG{p}{(}\PYG{n}{d}\PYG{p}{,} \PYG{n}{q1}\PYG{p}{,} \PYG{n}{d}\PYG{p}{,} \PYG{n}{q2}\PYG{p}{,} \PYG{n}{d}\PYG{p}{,} \PYG{n}{q1}\PYG{p}{,} \PYG{n}{d}\PYG{p}{,}\PYG{n}{q2}\PYG{p}{,}\PYG{n}{d}\PYG{p}{)} \PYG{p}{)}
\end{Verbatim}

Use {\hyperref[cpbd:twiss]{\code{twiss()}}} to find linear optics (Twiss) functions for given initial values

\begin{Verbatim}[commandchars=\\\{\}]
     \PYG{n}{tw0} \PYG{o}{=} \PYG{n}{Twiss}\PYG{p}{(}\PYG{p}{)}
     \PYG{n}{tw0}\PYG{o}{.}\PYG{n}{beta\PYGZus{}x} \PYG{o}{=} \PYG{l+m+mf}{5.}
     \PYG{n}{tw0}\PYG{o}{.}\PYG{n}{alpha\PYGZus{}x} \PYG{o}{=} \PYG{o}{\PYGZhy{}}\PYG{l+m+mf}{0.87}
     \PYG{n}{tw0}\PYG{o}{.}\PYG{n}{beta\PYGZus{}y} \PYG{o}{=} \PYG{l+m+mf}{2.1}
     \PYG{n}{tw0}\PYG{o}{.}\PYG{n}{alpha\PYGZus{}y} \PYG{o}{=} \PYG{l+m+mf}{0.96}
     \PYG{n}{tws} \PYG{o}{=} \PYG{n}{twiss}\PYG{p}{(}\PYG{n}{lat}\PYG{p}{,} \PYG{n}{tw0}\PYG{p}{)}
\end{Verbatim}

Find periodic Twiss solution

\begin{Verbatim}[commandchars=\\\{\}]
     \PYG{n}{tws} \PYG{o}{=} \PYG{n}{twiss}\PYG{p}{(}\PYG{n}{lat}\PYG{p}{)}
\end{Verbatim}

Find periodic Twiss solution with given longitudinal resolution (500 points)

\begin{Verbatim}[commandchars=\\\{\}]
     \PYG{n}{tws} \PYG{o}{=} \PYG{n}{twiss}\PYG{p}{(}\PYG{n}{lat}\PYG{p}{,} \PYG{n}{nPoints}\PYG{o}{=}\PYG{l+m+mi}{500}\PYG{p}{)}
\end{Verbatim}

Plot Twiss parameters

\begin{Verbatim}[commandchars=\\\{\}]
     \PYG{k+kn}{from} \PYG{n+nn}{pylab} \PYG{k+kn}{import} \PYG{o}{*}
     \PYG{n}{plot}\PYG{p}{(}\PYG{p}{[}\PYG{n}{t}\PYG{o}{.}\PYG{n}{s} \PYG{k}{for} \PYG{n}{t} \PYG{o+ow}{in} \PYG{n}{tws}\PYG{p}{]}\PYG{p}{,} \PYG{p}{[}\PYG{n}{t}\PYG{o}{.}\PYG{n}{beta\PYGZus{}x} \PYG{k}{for} \PYG{n}{t} \PYG{o+ow}{in} \PYG{n}{tws}\PYG{p}{]}\PYG{p}{)}
     \PYG{n}{plot}\PYG{p}{(}\PYG{p}{[}\PYG{n}{t}\PYG{o}{.}\PYG{n}{s} \PYG{k}{for} \PYG{n}{t} \PYG{o+ow}{in} \PYG{n}{tws}\PYG{p}{]}\PYG{p}{,} \PYG{p}{[}\PYG{n}{t}\PYG{o}{.}\PYG{n}{beta\PYGZus{}y} \PYG{k}{for} \PYG{n}{t} \PYG{o+ow}{in} \PYG{n}{tws}\PYG{p}{]}\PYG{p}{)}
\end{Verbatim}

Plot Twiss parameters in the lattice display

\begin{Verbatim}[commandchars=\\\{\}]
     \PYG{k+kn}{from} \PYG{n+nn}{ocelot.gui.accelerator} \PYG{k+kn}{import} \PYG{o}{*}
     \PYG{n}{plot\PYGZus{}opt\PYGZus{}func}\PYG{p}{(}\PYG{n}{lat}\PYG{p}{,} \PYG{n}{tws}\PYG{p}{)}
     \PYG{n}{show}\PYG{p}{(}\PYG{p}{)}
\end{Verbatim}


\subsection{Linear optics functions}
\label{cpbd:linear-optics-functions}\index{twiss() (built-in function)}

\begin{fulllineitems}
\phantomsection\label{cpbd:twiss}\pysiglinewithargsret{\bfcode{twiss}}{\emph{lat}\optional{, \emph{nPoints=None}}}{}
\end{fulllineitems}



\subsection{Matching}
\label{cpbd:matching}\index{match() (built-in function)}

\begin{fulllineitems}
\phantomsection\label{cpbd:match}\pysiglinewithargsret{\bfcode{match}}{\emph{lattice}, \emph{constarints}, \emph{variables}\optional{, \emph{start=0}}}{}
lattice a {\hyperref[cpbd:MagneticLattice]{\code{MagneticLattice}}} object

\end{fulllineitems}



\subsection{Tracking}
\label{cpbd:tracking}

\subsection{Elements}
\label{cpbd:elements}\index{MagneticLattice (built-in class)}

\begin{fulllineitems}
\phantomsection\label{cpbd:MagneticLattice}\pysigline{\strong{class }\bfcode{MagneticLattice}}
\end{fulllineitems}

\index{Drift (built-in class)}

\begin{fulllineitems}
\phantomsection\label{cpbd:Drift}\pysigline{\strong{class }\bfcode{Drift}}
\end{fulllineitems}

\index{Quadrupole (built-in class)}

\begin{fulllineitems}
\phantomsection\label{cpbd:Quadrupole}\pysigline{\strong{class }\bfcode{Quadrupole}}
\end{fulllineitems}

\index{Bend (built-in class)}

\begin{fulllineitems}
\phantomsection\label{cpbd:Bend}\pysigline{\strong{class }\bfcode{Bend}}\pysigline{\bfcode{same~as~SBend}}
\end{fulllineitems}

\index{SBend (built-in class)}

\begin{fulllineitems}
\phantomsection\label{cpbd:SBend}\pysigline{\strong{class }\bfcode{SBend}}
\end{fulllineitems}

\index{RBend (built-in class)}

\begin{fulllineitems}
\phantomsection\label{cpbd:RBend}\pysigline{\strong{class }\bfcode{RBend}}
\end{fulllineitems}



\subsection{Transfer maps}
\label{cpbd:transfer-maps}
Transfer maps define how the element map acts in tracking.
The default transfer map attachment scheme is as follows:
\begin{itemize}
\item {} 
Drifts, Quadrupoles, and bends have first order transfer maps

\item {} 
Sextupoles have a drift-kick-drift map

\end{itemize}


\subsection{API documentation}
\label{cpbd:api-documentation}\index{twiss() (in module ocelot.cpbd.optics)}

\begin{fulllineitems}
\phantomsection\label{cpbd:ocelot.cpbd.optics.twiss}\pysiglinewithargsret{\code{ocelot.cpbd.optics.}\bfcode{twiss}}{\emph{lattice}, \emph{tws0=None}, \emph{nPoints=None}}{}
\end{fulllineitems}

\index{match() (in module ocelot.cpbd.match)}

\begin{fulllineitems}
\phantomsection\label{cpbd:ocelot.cpbd.match.match}\pysiglinewithargsret{\code{ocelot.cpbd.match.}\bfcode{match}}{\emph{lat}, \emph{constr}, \emph{vars}, \emph{tw}, \emph{verbose=True}, \emph{max\_iter=1000}, \emph{method='simplex'}, \emph{weights=\textless{}function weights\_default at 0x2aedb4f16668\textgreater{}}, \emph{vary\_bend\_angle=False}, \emph{min\_i5=False}}{}
matching stuff

\end{fulllineitems}



\section{Synchrotron radiation (rad) module}
\label{radiation:synchrotron-radiation-rad-module}\label{radiation::doc}

\subsection{Overview}
\label{radiation:overview}
Synchrotron radiation from undulators and bending magnets


\section{Photon optics (optics) module}
\label{optics::doc}\label{optics:photon-optics-optics-module}

\subsection{Overview}
\label{optics:overview}
photon optics


\section{Machine interface (mint) module}
\label{mint::doc}\label{mint:machine-interface-mint-module}

\subsection{Overview}
\label{mint:overview}
Machine interface module


\section{Adaptors (adaptors) module}
\label{adaptors::doc}\label{adaptors:adaptors-adaptors-module}

\subsection{Overview}
\label{adaptors:overview}

\chapter{Indices and tables}
\label{index:indices-and-tables}\begin{itemize}
\item {} 
\emph{genindex}

\item {} 
\emph{modindex}

\item {} 
\emph{search}

\end{itemize}



\renewcommand{\indexname}{Index}
\printindex
\end{document}
